\documentclass{book}

\usepackage[utf8]{inputenc}
\usepackage[T1]{fontenc}
\usepackage[francais]{babel}

\title{Nidhogg 1.0}
\author{Félix \bsc{Foriel}}
\begin{document}

\maketitle
\chapter{Introduction}
J'ai créé le langage Nidhogg d'abord pour faire comprendre la programmation à mon père. Je venais juste d'apprendre le langage Perl et j'avais découvert ses
formidables possibilitées en matière de manipulation des chaînes de caractères, je décidais donc d'entreprendre la programmation d'un interpréteur Nidhogg/Python.
Il fut programmé en quelques jours et il rassemble les mots-clés de base du langage python, sa syntaxe coexiste en majuscule et en minuscule et les deux 
seront détaillées.
 
Une extension prenant en charge les modules Python sera peut-être un jour disponible. En attendant je vais vous détailler ici les différents mots-clés et 
leur signification dans l'ordre du script Perl.

\chapter{Les Mots-clés en majuscules}
\section{SINON SI elif }
  condition elif
\section{SINON else }
  condition else
\section{SI ECHOUE except }
  except du bloc try
\section{SI if }
  condition if
\section{ALORS : }
  le ":", then en basic
\section{TANT QUE while }
  boucle while
\section{POUR CHAQUE foreach }
  boucle spécifique à un tableau (pour chaque élément du tableau faire ...)
\section{POUR for }
  boucle for (pour chaque ... dans ... faire ...) 
\section{DANS in }
  in, s'utilise avec for (for ... in ...)
\section{CASSER break }
  sert à sortir d'une boucle
\section{CONTINUER continue }
  fait reboucler la boucle
\section{TOUT EST VRAI all }
  savoir si chaque élément d'une liste est vrai (renvoie True si vrai) 
\section{EST VRAI PARMI any }
  savoir si un élément d'une liste est vrai (renvoie True)
\section{VRAI True }
  booléen Vrai
\section{FAUX False }
  booléen Faux
\section{ECRIRE print }
  affiche du texte à l'écran
\section{ENTRER UNE CHAINE raw\_input }
  demande une saise à l'utilisateur qui sera convertie en chaîne de caractères
\section{ENTRER input }
  demande une saisie à l'utilisateur
\section{NOMBRE ALEATOIRE randint }
  donne un nombre aléatoire, s'utilise avec le module random
\section{NOMBRE int }
  type nombre
\section{DECIMALE float }
  type décimale
\section{CHAINE str }
  type chaîne de caractères
\section{EST EGAL A == }
  opérateur égal
\section{EST INFERIEUR OU EGAL A <= }
  opérateur inférieur ou égal
\section{EST SUPERIEUR OU EGAL A >= }
  opérateur supérieur ou égal 
\section{EST DIFFERENT DE != }
  opérateur différent
\section{EST INFERIEUR A < }
  opérateur inférieur
\section{EST SUPERIEUR A > }
  opérateur supérieur
\section{EGAL = }
  égal, s'utilise lors d'un calcul
\section{PLUS + }
  plus, s'utilise lors d'un calcul (a = 2+3)
\section{MOINS - }
  moins, idem
\section{PUISSANCE ** }
  puissance, idem
\section{DIVISE AVEC ENTIER PAR \ \  }
  diviser en récupérant un entier, idem
\section{MULTIPLIE PAR * }
  multiplier, idem
\section{DIVISE PAR \  }
  diviser, idem
\section{DIVISE AVEC RESTE PAR \% }
  diviser en récupérant le reste, idem
\section{LONGUEUR len }
  retourne la longueur d'une chaîne de caractères ou d'une liste
\section{SUPPRIMER del }
  supprime un élément d'une liste
\section{AJOUTER append }
  ajoute un élément à une liste
\section{TRIER LA LISTE sorted }
  trie une liste en créant une nouvelle liste 
\section{TRIER sort }
  trie une liste par ordre alphabétique ou numérique
\section{LISTE INVERSEE reversed }
  inverse une liste en en créat une autre
\section{INVERSER reverse }
  inverse les éléments d'une liste
\section{INDICE index }
  trouve l'indice d'un élément d'une liste
\section{ENLEVER remove }
  enlève un élément d'une liste
\section{GAMME range }
  donne une liste préremplie d'éléments
\section{DEPUIS from }
  sert à un importer un élément d'un module, s'utilise avec import (from ... import ...)
\section{IMPORTER import }
  sert à importer un élément d'un module, s'utilise avec from (idem)
\section{DEFINITION def }
  sert à définir une fonction (def ...( ... ) { ... })
\section{RENVOYER return }
  définit la valeur renvoyée par une fonction
\section{GLOBALE global }
  définit une variable globale au programme, réutilisable en dehors d'une fonction
\section{OUVRIR open }
  ouvre un fichier ( open("nom\_du\_fichier", "mode") )
\section{INSCRIRE write }
  écrit une ligne à l'intérieur d'un fichier
\section{LIRE UNE LIGNE readline }
  lit une ligne d'un fichier
\section{LIRE read }
  lit tout un fichier
\section{FERMER close }
  ferme un fichier
\section{TYPE type }
  renvoie le type d'une variable
\section{MELANGER LA LISTE zip }
  regroupe sous la forme d'un tuple les items d'une liste
\section{ADDITIONNER LA LISTE sum }
  additionne les valeurs d'un élément itérable 
\section{LISTE list }
  type liste
\section{COMPTER LES OCCURENCES I count }
  compte le nombre d'occurences d'une recherche dans une chaîne
\section{COMPTER LES OCCURENCES II count }
  compte le nombre d'occurences d'une recherche dans une liste
\section{ENUMERER enumerate }
  énumère les différents éléments d'une liste, est un itérateur 
\section{RECUPERER get}  retourne la valeur d'une clé donnée, s'utilise sur un dictionnaire
\section{COPIER copy }
  copie un dictionnaire, dans un ordre différent
\section{ABSOLUE abs }
  retourne la valeur absolue d'un nombre
\section{BINAIRE bin }
  convertit un nombre entier en chaîne de caractères binaires
\section{EST APPELLABLE callable }
  détermine si on peu appeller un objet, retourne True si vérifié
\section{CAPITALISER capitalize }
  formate une chaîne de caractère avec le format Xxxxx
\section{CHOIX choice }
  retourne aléatoirement une des valeurs d'une liste
\section{NOM DES STRUCTURES dir }
  retourne les noms de la structure d'un objet
\section{SE TERMINE PAR endswith }
  teste si une chaîne de caractère se termine par la chaîne demandée, retourne True si vérifié
\section{CALCULER eval }
  calcule une chaîne de caractères (eval("2+2") retourne 4)
\section{TROUVER find }
  donne la place de la première occurence de la chaîne demandée
\section{AIDER help }
  retourne des informations sur un élément demandé
\section{HEXADECIMAL hex }
  convertit un nombre en hexadécimal
\section{EST ALPHANUMERIQUE isalnum }
  vérifie que tous les caractères d'une chaîne sont alphanumériques, renvoie True si vérifié
\section{EST CARACTERIELLE isalpha }
  vérifie que tous les caractrèes d'une chaîne sont des lettres, idem
\section{EST CHIFFREE isdigit }
  vérifie que tous les caractères d'une chaîne sont des chiffres, idem
\section{EST MINUSCULE islower }
  vérifie que tous les caractères d'une chaîne sont en minuscule, idem 
\section{EST ESPACEE isspace }
  vérifie que la chaîne contient au moins un espace
\section{EST UN TITRE istitle }
  vérifie que la chaîne à un format titre, c'est à dire Xxxxx Xx Xxx Xxxx 
\section{TITRER title }
  formate une chaîne au format titre, idem
\section{EST MAJUSCULE isupper }
  vérifie que tous les caractères d'une chaîne sont en majuscules 
\section{JOINDRE join }
  transforme une liste en chaîne de caractères ("mon\_séparateur".join(ma\_liste))
\section{LOCALES locals }
  retourne un dictionnaire avec les valeurs des variables en cours 
\section{MINUSCULISER lower }
  met en minuscules une chaîne de caractères
\section{FAIRE map }
  execute une fonction sur chaque item d'une liste
\section{MAXIMUM max }
  retourne la valeur la plus élevée d'une liste 
\section{MINIMUM min }
  retourne la valeur la plus basse d'une liste
\section{ALEATOIRE random }
  retourne une valeur aléatoire
\section{REMPLACER replace }
  remplace un segment d'une chaîne de caractères par une autre
\section{ARRONDIR round }
  arrondi un nombre par le bas
\section{MELANGER shuffle }
  mélange aléatoirement une liste
\section{COMMENCER PAR startswith }
  vérifie qu'une chaîne de caractères commence par un préfixe indiqué
\section{SEPARER LES LIGNES splitlines }
  retourne une liste des lignes d'une chaîne
\section{SEPARER split }
  transforme une chaîne de caractères en liste ("ma\_chaîne".split("mon\_séparateur"))
\section{MAJUSCULISER upper }
  met en majuscules une chaîne de caractères
\section{ET and }
  vérifie qu'une condition ET une autre soient vérifiées
\section{OU or }
  vérifie qu'une condition OU une autre soient vérifiées
\section{PAS not }
  vérifie qu'une condition n'est pas vérifiée
\section{ESSAYER try }
  essaye un bout de code
\section{FINALEMENT finally }
  execute les instructions quelque soient les erreurs du bloc try
\section{CLASSE class }
  regroupe les attributs et les fonctions qui définissent un objet
\section{ELLE-MEME self }
  indique qu'il s'agit d'un attribut d'un objet
\section{INITIALISER \_\_init\_\_ }
  initialise un objet
\section{NOUVEAU \_\_new\_\_ }
  construit un objet
\section{OBJET object }
  type objet
\section{DICTIONNAIRE \_\_dict\_\_ }
  donne les valeurs de l'attribut d'instance
\section{DONNER yield }
  retourne une valeur sans la fin de la fonction
\section{AVEC with }
  execute du code avant et après un autre code
\section{COMME as }
  permet de récupérer une valeur dun with (with ... as ... : ...)
\section{N'EST PAS is not }
  opérateur différent 
\section{EST is }
  opérateur égal

\chapter{Les Mots-clés en minuscules}

\section{sinon si elif }
  condition elif
\section{sinon else }
  condition else
\section{si échoue except }
  except du bloc try
\section{si if }
  condition if
\section{alors : }
  le ":", then en basic
\section{tant que while }
  boucle while
\section{pour chaque foreach }
  boucle spécifique à un tableau (pour chaque élément du tableau faire ...)
\section{pour for }
  boucle for (pour chaque ... dans ... faire ...) 
\section{dans in }
  in, s'utilise avec for (for ... in ...)
\section{casser break }
  sert à sortir d'une boucle
\section{continuer continue }
  fait reboucler la boucle
\section{tout est vrai all }
  savoir si chaque élément d'une liste est vrai (renvoie true si vrai) 
\section{est vrai parmi any }
  savoir si un élément d'une liste est vrai (renvoie true)
\section{vrai true }
  booléen vrai
\section{faux false }
  booléen faux
\section{écrire print }
  affiche du texte à l'écran
\section{entrer une chaîne raw\_input }
  demande une saise à l'utilisateur qui sera convertie en chaîne de caractères
\section{entrer input }
  demande une saisie à l'utilisateur
\section{nombre aléatoire randint }
  donne un nombre aléatoire, s'utilise avec le module random
\section{nombre int }
  type nombre
\section{décimale float }
  type décimale
\section{chaîne str }
  type chaîne de caractères
\section{est égal à == }
  opérateur égal
\section{est inférieur ou égal à <= }
  opérateur inférieur ou égal
\section{est supérieur ou égal à >= }
  opérateur supérieur ou égal 
\section{est différent de != }
  opérateur différent
\section{est inférieur à < }
  opérateur inférieur
\section{est supérieur à > }
  opérateur supérieur
\section{égal = }
  égal, s'utilise lors d'un calcul
\section{plus + }
  plus, s'utilise lors d'un calcul (a = 2+3)
\section{moins - }
  moins, idem
\section{puissance ** }
  puissance, idem
\section{divisé avec entier par \ \  }
  diviser en récupérant un entier, idem
\section{multiplié par * }
  multiplier, idem
\section{divisé par \  }
  diviser, idem
\section{divisé avec reste par \% }
  diviser en récupérant le reste, idem
\section{longueur len }
  retourne la longueur d'une chaîne de caractères ou d'une liste
\section{supprimer del }
  supprime un élément d'une liste
\section{ajouter append }
  ajoute un élément à une liste
\section{trier la liste sorted }
  trie une liste en créant une nouvelle liste 
\section{trier sort }
  trie une liste par ordre alphabétique ou numérique
\section{liste inversee reversed }
  inverse une liste en en créat une autre
\section{inverser reverse }
  inverse les éléments d'une liste
\section{indice index }
  trouve l'indice d'un élément d'une liste
\section{enlever remove }
  enlève un élément d'une liste
\section{gamme range }
  donne une liste préremplie d'éléments
\section{depuis from }
  sert à un importer un élément d'un module, s'utilise avec import (from ... import ...)
\section{importer import }
  sert à importer un élément d'un module, s'utilise avec from (idem)
\section{définition def }
  sert à définir une fonction (def ...( ... ) { ... })
\section{renvoyer return }
  définit la valeur renvoyée par une fonction
\section{globale global }
  définit une variable globale au programme, réutilisable en dehors d'une fonction
\section{ouvrir open }
  ouvre un fichier ( open("nom\_du\_fichier", "mode") )
\section{inscrire write }
  écrit une ligne à l'intérieur d'un fichier
\section{lire une ligne readline }
  lit une ligne d'un fichier
\section{lire read }
  lit tout un fichier
\section{fermer close }
  ferme un fichier
\section{type type }
  renvoie le type d'une variable
\section{mélanger la liste zip }
  regroupe sous la forme d'un tuple les items d'une liste
\section{additionner la liste sum }
  additionne les valeurs d'un élément itérable 
\section{liste list }
  type liste
\section{compter les occurences I count }
  compte le nombre d'occurences d'une recherche dans une chaîne
\section{compter les occurences II count }
  compte le nombre d'occurences d'une recherche dans une liste
\section{énumérer enumerate }
  énumère les différents éléments d'une liste, est un itérateur 
\section{récupérer get}  retourne la valeur d'une clé donnée, s'utilise sur un dictionnaire
\section{copier copy }
  copie un dictionnaire, dans un ordre différent
\section{absolue abs }
  retourne la valeur absolue d'un nombre
\section{binaire bin }
  convertit un nombre entier en chaîne de caractères binaires
\section{est appellable callable }
  détermine si on peu appeller un objet, retourne true si vérifié
\section{capitaliser capitalize }
  formate une chaîne de caractère avec le format xxxxx
\section{choix choice }
  retourne aléatoirement une des valeurs d'une liste
\section{nom des structures dir }
  retourne les noms de la structure d'un objet
\section{se termine par endswith }
  teste si une chaîne de caractère se termine par la chaîne demandée, retourne true si vérifié
\section{calculer eval }
  calcule une chaîne de caractères (eval("2+2") retourne 4)
\section{trouver find }
  donne la place de la première occurence de la chaîne demandée
\section{aider help }
  retourne des informations sur un élément demandé
\section{hexadécimal hex }
  convertit un nombre en hexadécimal
\section{est alphanumérique isalnum }
  vérifie que tous les caractères d'une chaîne sont alphanumériques, renvoie true si vérifié
\section{est caractérielle isalpha }
  vérifie que tous les caractrèes d'une chaîne sont des lettres, idem
\section{est chiffrée isdigit }
  vérifie que tous les caractères d'une chaîne sont des chiffres, idem
\section{est minuscule islower }
  vérifie que tous les caractères d'une chaîne sont en minuscule, idem 
\section{est espacée isspace }
  vérifie que la chaîne contient au moins un espace
\section{est un titre istitle }
  vérifie que la chaîne à un format titre, c'est à dire xxxxx xx xxx xxxx 
\section{titrer title }
  formate une chaîne au format titre, idem
\section{est majuscule isupper }
  vérifie que tous les caractères d'une chaîne sont en majuscules 
\section{joindre join }
  transforme une liste en chaîne de caractères ("mon\_séparateur".join(ma\_liste))
\section{locales locals }
  retourne un dictionnaire avec les valeurs des variables en cours 
\section{minusculiser lower }
  met en minuscules une chaîne de caractères
\section{faire map }
  execute une fonction sur chaque item d'une liste
\section{maximum max }
  retourne la valeur la plus élevée d'une liste 
\section{minimum min }
  retourne la valeur la plus basse d'une liste
\section{aléatoire random }
  retourne une valeur aléatoire
\section{remplacer replace }
  remplace un segment d'une chaîne de caractères par une autre
\section{arrondir round }
  arrondi un nombre par le bas
\section{mélanger shuffle }
  mélange aléatoirement une liste
\section{commence par startswith }
  vérifie qu'une chaîne de caractères commence par un préfixe indiqué
\section{séparer les lignes splitlines }
  retourne une liste des lignes d'une chaîne
\section{séparer split }
  transforme une chaîne de caractères en liste ("ma\_chaîne".split("mon\_séparateur"))
\section{majusculiser upper }
  met en majuscules une chaîne de caractères
\section{et and }
  vérifie qu'une condition et une autre soient vérifiées
\section{ou or }
  vérifie qu'une condition ou une autre soient vérifiées
\section{pas not }
  vérifie qu'une condition n'est pas vérifiée
\section{essayer try }
  essaye un bout de code
\section{finalement finally }
  execute les instructions quelque soient les erreurs du bloc try
\section{classe class }
  regroupe les attributs et les fonctions qui définissent un objet
\section{elle-même self }
  indique qu'il s'agit d'un attribut d'un objet
\section{initialiser \_\_init\_\_ }
  initialise un objet
\section{nouveau \_\_new\_\_ }
  construit un objet
\section{objet object }
  type objet
\section{dictionnaire \_\_dict\_\_ }
  donne les valeurs de l'attribut d'instance
\section{donner yield }
  retourne une valeur sans la fin de la fonction
\section{avec with }
  execute du code avant et après un autre code
\section{comme as }
  permet de récupérer une valeur dun with (with ... as ... : ...)
\section{n'est pas is not }
  opérateur différent 
\section{est is }
  opérateur égal
\end{document}